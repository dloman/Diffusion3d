\documentclass[11pt,twocolumn]{article}

\title{\textbf{CS 652 Final Project }}
\author{Daniel Loman}
\date{6/18/2012}
\usepackage[margin=0.5in]{geometry}
\begin{document}
\maketitle
\begin{abstract}
Conway's Game of Life is the classic example of emergence in cellular automata. Presented here is a 2D and 3D expansion of the original version of Conway's Game of Life. Extension into 3 spatial dimensions as well as an extension of the 1 or 0 based neuron approach to a numerical scale ranging from 1 to 10. During this study we witnessed distinct emergent behaviors... when completed...
\end{abstract}
\section{Introduction}
Conway's Game of Life, further refereed to as GoL, is a cellular automata based game based upon the mathematical problems presented by John Von Neumann. The GoL provides an simple mechanism for the study of emergence in self-organization.
Extending the GoL to 3 spatial dimensions and increasing the complexity of each individual automata state, allows a significant increase in the complexity of the interactions we are able to model. 
Allowing for the simulation of diffusion, whether that be diffusion of chemical reactions or heat flow.  
\subsection{Cellular Automata}
 Cellular automata modeling techniuqe based on modeling a manifold 
 using a finite number of discrete states. 
 Each automata has a set of additionally cells commonly refered to as its neighbors. 
 This neighborhood of cells are the set of cells residing some specific minimum spatial distance from the cell itself. 
 Cellular automata state transformation are dependent on the state of thier neighbors and specific set of rules that the cellular automata is operating on. 
 This respect is where cellular automata differ from coupled map lattice, state values are entirley dependent on inital conditions, and neighborhood state values.
 Traditionally the structure of cellular automata states are constructed using the neuron model, or the states can be either 'on' or 'off'. 
 More Complex automata can also be modeled with state complexity significantly more complex then 1s and 0s.
 \subsection{Diffusion}
 Diffusion is transportation mechanism that doesn not require bulk movement. 
 Molecular diffusion is consider the result of the random walk powered by thermal energy. 
 Diffusion is most often model using vector wave equations. Vector calculous models of diffusion allow for the modeling of an infinite number of particles qucikly and computatinally less intensive. 
 Discretization of diffusion allows for more a more fine grained approach, at the cost of an increase in compuational complexity.
\section{Experimental Background}
Diffusion


\end{document}
